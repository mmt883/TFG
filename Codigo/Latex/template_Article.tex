\documentclass[]{article}

%opening
\title{Reconocimiento de Actividad Humana mediante Automatas Finitos}
\author{Miguel Martín Turrión}

% Definir el entorno "resumen"
\newenvironment{resumen}{
	\begin{quote}
		\textbf{Resumen:} \\
	}{
	\end{quote}
}


\begin{document}

\maketitle

\begin{resumen}
Texto de ejemplo
\end{resumen}

\begin{abstract}
Texto de ejemplo
\end{abstract}

\section{Introducción(Descripcion del problema, human activity recognition,...)}
                Texto de ejemplo
\section{Descripción del dataset, concurso}      
El dataset del laboratorio inteligente del UJAml ha sido tomado por cuatro fuentes de datos distintas durante un periodo de diez dias donde un habitante ha realizado 246 acciones pertenecientes a 24 tipos distintos de actividades. \\

\noindent Las cuatro fuentes de datos proporcionadas son las siguientes:

\begin{itemize}
	\item Secuencias de datos generadas por treinta sensores binarios
	
	\item Información de proximidad entre un reloj inteligente llevado por el habitante y un conjunto de 15 (Bluetooth Low Energy (BLE) beacons) repartidos en el laboratorio inteligente UJAml
	
	\item Acceleracion generada por el reloj inteligente
	
	\item Suelo inteligente con 40 modulos que proporcionan informacion espacial
\end{itemize}
El conjunto de actividades ha sido divido en dos partes:
\vspace(3pt)
\begin{itemize}
	\item[1º]\textbf{Parte de Entrenamiento o Training Set:} Conjunto de entrenamiento conocido formado durante siete dias que contiene 169 acciones
	
	\item[2º]\textbf{Parte de Comprobación o Test Set:} Conjunto de comprobación desconocido formado durante tres dias que contiene 77 acciones
\end{itemize}

	



\section{Metodologia}
\subsection{grafo mañana, tarde (automata con pesos)}
\subsection{grafo actividad $\rightarrow$ expresion Regular (algoritmo rpni)}
\subsection{prediccion}
\section{Implementacion(Ejemplos concretos, explicacion de algoritmo rpni, casos problematicos)}
\section{Resultados y fine\-turing)}
\section{Conclusiones finales}


\end{document}
