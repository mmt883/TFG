\documentclass[]{article}

%opening
\title{Reconocimiento de Actividad Humana a traves de Automatas Finitos}
\author{Miguel Martín Turrión}

% Definir el entorno "resumen"
\newenvironment{resumen}{
	\begin{quote}
		\textbf{Resumen:} \\
	}{
	\end{quote}
}


\begin{document}

\maketitle

\begin{resumen}
Este trabajo aborda el problema de la identificacion de actividades humanas en un entorno ubicuo, donde los datos se recopilan de una amplia variedad de fuentes. En nuestro enfoque, despues de filtrar las entradas ruidosas de los sensores, aprendemos los patrones de comportamiento del usuario y los patrones de los sensores de las actividades mediante la construccion de automatas finitos ponderados y expresiones regulares, respectivamente, e inferimos la posicion del habitante para cada actividad a traves de la distribucion de frecuencia de los datos de los sensores de piso. Finalmente, analizamos los resultados de la prediccion de esta estrategia, que obtiene una precision del 90.65\% en los datos de prueba.
\end{resumen}

\begin{abstract}
This work addresses the problem of human activity identification in an ubiquitous
environment, where data is collected from a wide variety of sources. In our approach, after filtering
noisy sensor entries, we learn user’s behavioral patterns and activities’ sensor patterns through
the construction of weighted finite automata and regular expressions respectively, and infer the
inhabitant’s position for each activity through frequency distribution of floor sensor data. Finally, we
analyze the prediction results of this strategy, which obtains 90.65\% accuracy for the test data.
\end{abstract}

\section{}

\end{document}
